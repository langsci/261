\documentclass[output=paper]{langsci/langscibook} 
\author{Ellen B. Basso \affiliation{University of Arizona}} 

\markuptitle{Interpersonal alignments and epistemic marking in Kalapalo (Southern Carib, Brazil)}{Interpersonal alignments and epistemic marking in Kalapalo (Southern Carib, Brazil)}
\shorttitlerunninghead{Interpersonal alignments and epistemic marking in Kalapalo}
%-- Syposium on Evidentiality, egophoricity, and engagement: descriptive and typological perspective

% \title{\texorpdfstring{Word formation and word history:\\ The case of
%  \textsc{capitalist} and \textsc{capitalism}}{Word formation and word
% history:  CAPITALIST and
% CAPITALISM}}

% \renewcommand{\lsCollectionPaperFooterTitle}{Word formation and word
% history: The case of  \noexpand\textsc{capitalist} and
% \noexpand\textsc{capitalism}}

 

\abstract{}

\begin{document}
\maketitle

\section{Introduction}\label{s:eb1} 

The idea of “interpersonal relations” is used in this paper to refer to a language-focused developmental process involving emergent patterns of communication and understanding in the interaction of distinctively different people within a community. It was probably used anthropologically by the modernist Edward \citeauthor{Sapir1932} (\citeyear{Sapir1932}; \citealt{Basso1998}).  Sapir repeatedly noted in his lectures at Yale that the interaction of people of different personalities and their influence on the thought and action of a community has consequences with many historical implications (\citealt[204]{Sapir1993}). When people respond to one another during conversational interaction as they receive information about the worlds of others, Sapir noted, they participate in contexts wherein  judgments or evaluations of propositions are being made, often while speakers are trying to make decisions and plans for future action.  More recently,   John \citeauthor{DuBois2004}’ technical use of “alignment” refers to such ongoing activity   ‘in which two participants in dialogical interaction ...converse in varying degrees’  in taking a stance (\citeyear[22--23]{DuBois2004}).  Evidential and epistemic markers contribute to a speaker’s evaluation, self-positioning, and alignments with other subjects concerning the sharing of knowledge and evaluation of the epistemic object or proposition. (\citealt[24]{DuBois2007}).  Such stance enactments are central to the pragmatic functioning  of sociality and everyday power (\citealt[5--6]{Ameka2004}).   
%\isi{finality}

In this paper I discuss stance enactments as they occur among speakers of Kalapalo,  a Southern Cariban agglutinative language currently spoken by about 630 people in the Alto Xingu region of central Brazil (\citealt{Guerreiro2015}). In Kalapalo, a large number of epistemic markers are central to the pragmatic functioning of social life.  My interest in this  epistemic marking has grown out of my earlier anthropological work on Kalapalo narrative discourse (\textit{akiñatunda}) , a dialogical practice structured in large measure by the interaction between a narrator (\textit{akiña otoi})  and a ‘listener-responder’ (\textit{etuitsofo}) (\citealt{Basso1985}; \citeyear{Basso1986}; \citeyear{Basso1987}; \citeyear{Basso1995}). Analyses of recorded narrated speech events assisted my understanding of the use of the epistemic features.  In this material, dialogicality exists in the many narrated quoted conversations in which occur emergent processes of challenge, resistance, debate, deception, and the negotiation of meaning.  Emotional events such as respect and endearment, as well as anger, shame, lust, greed, and envy are important for descriptions of protagonists  and it is through their quoted speech that we can see how these emerge from social interaction (\citealt{Basso2007}).  A narrator’s authority to speak of imaginary or historical subjects is an adventure in language, a kind of critical social commentary that may not be tolerated in other public situations.  Presentation of sociohistorical and mythological realities include fantasies regarding human bodily processes and  tricksters’ appetite for subversion,  as well as historical memories of women who have been abducted and who learned to live in a different kind of society,  and the personal adventures of warriors and shamans.  The materials presented here include quoted conversations (often, interspersed with narrators’ commentary), as well as from my own conversations with Kalapalo, and from the ritual speech of hereditary leaders (\citealt{Basso2009}; see also \citealt{Franchetto1983}, \citeyear{Franchetto2000}).
 
In Kalapalo there is a set of six hearsay evidential markers (EV) mainly used in narrative and in quoted discourse (Table \ref{tab:eb1});  %TODO EV and EXP?
eight initial position epistemic expressives (EXP) which directly mark first person judgment  and sometimes 1$^{st}$ person evidence (Table \ref{tab:eb2})\footnote{Another set of expressives are affective (marking fear, pain, sensory pleasure, disgust, grief, frustration). Their use by Kalapalo speakers tends to create a synergy of sentiment with listeners. The epistemic expressives are both interpersonal as well as cognitive and introspective in semantic function as shown in Table \ref{tab:eb2}. Both sub-sets of the expressive word class are in the main used by speakers to introduce further descriptive commentary.}
and the  large set (28) %TODO what does that number refer to?
of second position epistemic clitics or free particles (EM) that I have ordered into six pragmatic sets; EM numbers correspond to the full set listed in the Appendix\footnote{Due to the large number of EM markers and complex semantic features, some of which (e.g., assertion, inferential, negation) may be repeated in the use of several forms, I have used numbers rather than semantically informative glosses. Full descriptions occur in the examples and the Appendix.}. 
 
%table 1
\begin{table}
\begin{tabularx}{\textwidth}{lL{3.5cm}L{3.5cm}}
\hline
 & \textbf{Meanings} & \textbf{Discourse contexts}\\
\hline
\textit{	tï	}	&	narrative hearsay	&	narrative  line or quotative	\\
\textit{	-fi-	}	&	appears on the quotative only, marking  a neutral or vague source of information	&	can be used as a boundary marker separating conversation from authoritative narrative speech	\\
\textit{	kila	}	&	recent knowledge transmitted by speaker,  received from a third person	&	comment on 3$^{rd}$ p. speech act	\\
\textit{	kita	}	&	historical knowledge given to listener in another context	&	event description (verbal clause)	\\
\textit{	nïgï i-feke	}	&	quotative, perfective	&	‘X said’ (conversational)	\\
\textit{	ta i-feke	}	&	quotative, continuative	&	‘X saying’ (response)	\\
\hline
\end{tabularx}
\caption{Evidential suffixes (EV) and Evidential strategies (quotatives)}
\label{tab:eb1}
\end{table}

% %table 2
\begin{table}
\begin{tabularx}{\textwidth}{lL{3.5cm}L{3.5cm}}
\hline
\textbf{Epistemic expressives} & \textbf{Meanings} & \textbf{Translation or other functions}\\
\hline
\textit{	u:um	}	&	insight, planning based on inference	&	‘I’m thinking’	\\
\textit{	ah	}	&	assertive affirmative, evidential	&	‘the fact is’	\\
\textit{	koh	}	&	unknown	&	‘I don’t know’.	\\
\textit{	eh	}	&	affirmation	&	‘yes’	\\
\textit{	eh he	}	&	acknowledgment of speech act	&	‘I hear you’	\\
\textit{	he	}	&	acknowledgment (reduced form)	&	‘you’re right’	\\
\textit{	ta:	}	&	rhetorical negation	&	‘how/why not’	\\
\textit{	akah	}	&	mirative negative, frustrative	&	‘oh my’	\\
\textit{	ekï	}	&	temporarily forgotten name	&	‘um’	\\
\textit{	u:m ma	}	&	an attempt to try to understand	&	‘I can’t be sure’	\\
\textit{	ah u:m	}	&	certainty of imaginative understanding	&	‘I’m sure’	\\
\hline
\end{tabularx}
\caption{Epistemic Expressives (EXP)}
\label{tab:eb2}
\end{table}
\clearpage

Kalapalo epistemic markers appear at the end of an adverbial phrase, a nominal phrase or clause, or a converbal or lexical verbal clause. There is rarely any vowel reduction; excepting with the morpheme \textit{aka}, and no epistemic clitic are vowel initial.  The morpheme \textit{muk$^w$e} may show elision when followed by a vowel initial phonological word.  Nonetheless, most epistemic morphemes function as clitics insofar as they are phonologically bound through (second syllable) stress patterning to their host constructions, thereby playing a key role in phonological word construction.  Epistemic markers have the ability to be compounded with one another, and with other types of clitics (\citealt{Basso2014}), as seen in examples \ref{ex:eb6}, \ref{ex:eb35b}, \ref{ex:eb37}, \ref{ex:eb41a},  \ref{ex:eb52}, \ref{ex:eb63c}.\footnote{Similar morphemes have been described in the pioneering work on Northern Cariban languages  by  B.J. \cite{Hoff1986}, Eithne \cite{Carlin2004}; and  Sergio \cite{Meira1999}.  Northern  Cariban epistemic clitics usually appear after the first element in a clause (Wackernagel’s position); the scope of the clitic including the entire clause (\citealt{Hoff1986}).}
 
 
A number of the Kalapalo parameters seem to closely match those specified by B.J. \cite{Hoff1986} for Surinam Carib (Kari’na) and also Bruna \citeauthor{Franchetto}’ s (n.d.) discussion of  (so-called modal) ‘marker’s of true speech’ in the closely related Alto Xingu Kuikuro language.  However, with one exception, I use different language to describe these features in keeping with my emphasis on person-person deixis and dialogicality in stance processes.  What \citeauthor{Hoff1986} calls “grade” (strong-weak-lacking); “speaker’s attitude”, “appeals on speaker”, “change of grade” and “co-existence with supporting or conflicting evidence” are clearly semantic parameters found in Kalapalo epistemic marking.  \citeauthor{Hoff1986}’s specification of an underlying contrast in the Surinam Carib particle set between the origins of “extraspective” and  “introspective” evidence  is also an evidential feature in Kalapalo, an this can be understood with reference to Ferdinand \citeauthor{Haan2001}’s (\citeyear{Haan2001}) approach that allows for a pragmatic description subsuming evidential and epistemic features.  \citeauthor{Haan2001} distinguishes between two deictic categories relevant to evidentiality: in the first, the speaker is separated (or separates themselves) from the action being described; in the second, the speaker includes themself in the description of the action.
 In fact, these make sense for Kalapalo epistemic marking, when the stance object is foregrounded in the discourse segment  (particularly in Sets A and B). 
But there is also the third person marked both as a logophoric interlocutor and as an epistemic object.  There are four epistemic markers which can be used to describe logophoricity of several kinds: \textbf{\textit{kafa}} Em17, weak but positive; \textbf{\textit{kato}} Em18, troublesome ; \textbf{\textit{nafa}} Em20, an animate but non-human response; \textbf{\textit{kalaka}} Em23, reflexive desire.  Alignment (or ‘engagement’) is yet another deictic field,   involving a speaker referencing an other subject’s inclusion in the sphere of epistemic stance (as in Sets E and F). 
These involve participant frameworks relevant to Kalapalo epistemic markers in which the intersubjective relation is foregrounded and the object-subject relation recedes (though it is not entirely absent from discursive commentary). 

While epistemic modality in the sense of a “scale” or “grade” is a feature of Kalapalo epistemology, there are important non-modal features that emerge as speakers evaluate information, including a) participation (or not)  of the speaker and listener and a third person in a cooperative epistemic context, b) changes or correctives due to receipt of new information; c) mirativity (due to new and surprising information) and incredulity (an extreme skepticism or unwillingness to believe); d) counter-factive or contra-spective wishing or hoping, and e) acceptance or rejection of a conjoint project.  I show in examples how the four logophoric markers mark responses to different kinds of experiences, speech, or desires and plans of someone other than the interlocutors:  
Although I follow earlier writers (\citealt{Aikhenvald2004}, \citealt{Haan1999}, \citeyear{Haan2001}; \citealt{NuckollsLev2012}), in making a distinction between evidentiality (EV) and epistemic markers (EM) as shown in the preceding tables,   Kalapalo EV, EXP and EM are often combined in discourse as many of the examples will show; and some EM also mark types of evidentiality (first hand experience; inference (or lack of direct speech evidence) regarding a second or third person’s thoughts and plans); these are not seen in the EV set.  Past tense semantics occurs with the use of \textbf{\textit{wãke}} EM1 and \textbf{\textit{nipa}} EM 24, of interest as there is no past tense marking on the verb.  In addition four EM (\textbf{\textit{kiŋi}}, plural \textbf{\textit{kiŋi-ni}} EM11, \textbf{\textit{pile}} EM15, tiki EM19,  \textbf{\textit{tima}} EM28)  have negation meanings.  What follows is a brief listing of the six epistemic sets sorted according to pragmatic and semantic features. These sets are discussed further with examples in Sections \ref{s:eb2}-\ref{s:eb4}, followed by my final observations in section \ref{s:eb5}.  Names of original narrators and location of my research recordings are given following the examples.	
	
Outline of Epistemic Sets
\begin{itemize}
	\item Sets A-D mark a focus upon the speaker’s evaluation of a proposition and ‘positioning’%4 %TODO FN4?
	\begin{itemize}
		\item Set A: knowledge has been received internally; the speaker is a participant in the propositional context. 
		\item Set B: knowledge is received externally; the speaker is not a participant in the propositional context but must infer or deduce from this external evidence.     
	\end{itemize}    
	\item Sets C and D: involve subjective ‘positioning’ used after a speaker has received new information
	\begin{itemize}
		\item Set C involves self-correctives 
		\item Set D  concerns counter-factive or contra-spective acknowledgment,  wishing or hoping’
	\end{itemize}
	\item Sets E and F concern the speaker’s marking of epistemic alignment, that is, participation in a shared or in two cases, disputed propositional context. 
	\begin{itemize}
		\item Set E marks shared information; one negative marks refusal to share information
		\item Set F marks speaker’s proposal or rejection of a conjoint project. In this set there is a marker with negative meaning used when the speaker refuses to participate or to be blamed in such a context.
	\end{itemize}
\end{itemize}

\section{Internal versus extenal contrasts in Sets A and B}\label{s:eb2}

Sets A and B are similar to Berend \citeauthor{Hoff1986}’s (\citeyear{Hoff1986}) specification of an underlying contrast in the Northern Cariban Kari’na (or Surinam Carib) particle set between the origins of two types of evidence.  The first is “introspective”, “inner world of the speaker”; evidence in the mind of the speaker; private knowledge.  The second is ‘extraspective’ evidence from the “outside world”, that is, evidence external to the speaker’s experiences; or public knowledge. 

\subsection{Set A}   

Knowledge is ‘internal’, part of the speaker’s experience. There are six members of this set.  These markers seem to reference an interloctor’s thoughts or ideas, rather than direct speech so contrastwith the hearsay EV group.

\subsubsection{\textit{wãke} EM1} 
With this frequently used marker, the speaker asserts a (usually distant) past experience in which knowledge or evidence has been acquired. The strong assertion is often seen in the speaker’s repetition of the marker after each clause.  In the example, upon hearing that his friend is the lone survivor of a massacre, the speaker declares that was why he didn’t join them earlier, expecting to have to avenge them all.

%TODO glossing 
\begin{exe}
	\ex \label{ex:eb1}
	\gll ege-tomi=dye-fa \textbf{wãke} u-te-lï \textbf{wãke}, o-pi-ñï-ko-i u-its-ani, u-ki-lï \textbf{wãke}\\
	PDEM-PURP=SS-TOP EM1 1-go.away-PNCT EM1 2-avenge-AN-PL-COP 1-EX-FUT.1, 1-utter-PNCT EM1\\
	\trans ‘For this purpose \textbf{I remember some time ago} I said, ‘I will be the one to go away as your avenger.’ (said by Kudyu at Aifa, 1980)
\end{exe}

\subsubsection{\textit{tifa} EM2}
This marks affirmation of the interlocutor’s current understanding of  (or thoughts about )  the speaker’s recently spoken ideas. This marker may be based on the hearsay EV \textbf{\textit{ti}-}.  See example (\ref{ex:eb41a}) where EM2 contributes to the interlocutors belief that what he is saying is true.   Most examples apear to be referencing a 2$^{nd}$ or 3$^{rd}$ person’s thoughts rather than speech. 
%a.
The following example marks an exclusive plural person’s acceptance. 


\begin{exe}
	\ex \label{ex:eb2}
	\gll tis-eti-dyi-pïgï-iña=\textbf{tifa} ti-ŋifa-nïgï.\\
	1+3-come.out-ITR-VPE-BEN=\textbf{EM2} 1+3-teach-PS\\
	\trans ‘\textbf{You are right} to think that on behalf of our offspring (i.e., ‘those who come out of us’), is what we teach.' (said by Kudyu at Aifa, 1979)
\end{exe}

%b. 
Here, the speaker uses both EM2 in the nominalization part of the construction and  \textbf{\textit{taka}} EM22 \emph{agreement with someone else’s description of their experience} in the bracketed adverbial clause;  EM22 references his mother’s just-stated dismay at seeing his reddened eyes; the EM2 \textbf{\textit{tifa}} references the speaker’s acknowledgment of what she thought of the events being described. 

\begin{exe}
	\ex \label{ex:eb3}
	\gll (afïtï=\textbf{taka} igei-ufuŋu) ta-ŋo-fïŋï-\textbf{tifa}... tisuge, afïtï\\
	denial=EM22 IDEM-unlike DIS-NLOC-resemble-EM2 1+3, denial.\\
	\trans ‘No, (=\textbf{because of what you saw}) you must know there’s got to be another kind of place where we can live, no.' (said by Kudyu at Aifa, 1979)
\end{exe}

%c.
Here the speaker confirms a 2p.’s understanding of a past event.  \textbf{\textit{wãke}} EM1 is used to assert the speaker’s own participation. Because of the group context,  I have used a first person plural translation. 

\begin{exe}
	\ex \label{ex:eb4}
	\gll ule-tse-ngugi=\textbf{tifa}=\textbf{wãke}, ngikogo e-nïgï \textbf{wãke}. aifu-pe-fa.\\
	AFR-be-interrupted=EM2=EM1 {fierce people} come-PERF EM1. end-ESS-PTP\\
	\trans ‘\textbf{We saw} he was interrupted \textbf{just as you thought he would be} when he started things, \textbf{we saw} the fierce people come. It was all over for him.’
\end{exe}

%(3) 
\subsubsection{\textit{laka} EM3} 
This marks the speaker’s  very weak inference, puzzlement, or a strong inability to understand the participatory context. 
%a.  
This example includes both EM3 and the EXP \textbf{\textit{uum}}, \emph{referencing thought of an imaginative or inferential nature}. In \citeauthor{Haan2001}’s deictic model (\citeyear{Haan2001}), the speaker has put himself into the sphere of action.

\begin{exe}
	\ex \label{ex:eb5}
	\gll \textbf{u:um} aŋi=\textbf{laka} ukuge ele-i ukuge\\
	\textbf{EXP} result=\textbf{EM3} {human being} DEM-COP {human being}\\
	\trans ‘I suppose that could have been human, a human being’. (said by Tufule at Aifa, 1982)
\end{exe}

%b. 
In a trickster story,  the speaker is surprised that he has already been told not to eat some fish parts, although he has already done so. The marker \textit{seku} following the verb \textit{te}-,’ eat’ is the mirativity marker compounded with EM3.  

\begin{exe}
	\ex \label{ex:eb6}
	\gll te-ŋe=seku=\textbf{laka} a-nïgï u-feke=lefa\\
	  eat.flesh-TRNS=MIR=EM3 EQS-PFV 1-ERG=SEQ\\
	\trans ‘I don’t think I was already told not to eat flesh!’ (said by Tufule at Aifa,1982)
\end{exe}

%(4) 
\subsubsection{\textit{ma} EM4}
Used as a clitic, this dubitative form references the speaker’s uncertainty due to lack of knowledge.  It is used mainly, but not exclusively, with interrogative forms. It may also serve as an epistemic neutral marker (see also conversational examples (\ref{ex:eb20})--(\ref{ex:eb23}), (\ref{ex:eb24})--(\ref{ex:eb26}), (\ref{ex:eb45c}), (\ref{ex:eb49c}), (\ref{ex:eb55c})). In the example, a narrator questions how someone in his story could put a large fish inside a small flute:

\begin{exe}
	\ex \label{ex:eb7}
	\gll ta-me=\textbf{ma} kuluta atati tu-i-ŋalï i-feke\\
	CONT-FACS=EM4 flute inside REF-put-REV 3-ERG\\
	\trans ‘How could he have put it back inside something like that flute?’ (said by Kambe at Aifa 1979)
\end{exe}

%(5) 
\subsubsection{\textit{maŋa} EM5} 
The speaker, while enunciating a proposition, at the same time denies any possibility of its occurrence; this form is used to express incredulity, often with the rhetorical (y/n) question prefix \textbf{\textit{tï}}. (‘I can’t believe you’re asking/saying (X) as you and I both well know the answer’.)

%a. 
The envious Trickster declares he will make the same kinds of rare and beautiful things given to his younger brother.


\begin{exe}
	\ex \label{ex:eb8}
	\gll \textbf{u:um} tï-kïtsï=\textbf{maŋa} t-iŋuG-isi Taugi ki-lï\\
	EXP RQ-ugly=EM5 REF-make-ADV (name) utter-PNCT\\
	\trans ‘Who thinks this is hard for me to make?/ I can only make it badly?” (=”This \textbf{is} easy for me to make”), Taugi spoke.’ (said by Tufule at Aifa, 1979)
\end{exe}

%b. 	
A leader’s ritual communication is often filled with this kind of ironic restraint, a kind of respectful devaluation of the work of his own messengers.  There is a feeling of the speaker’s modesty enacted by his disclaimer. 


\begin{exe}
	\ex \label{ex:eb9}
	\gll tï-kaïŋa-fïŋï=\textbf{maŋa} Ø-atsa-ki-lï a-tehe=gele-fa \textbf{wãke}\\
	RQ-DEST-resemble=EM5 3-run-TRNS-PNCT EQS-PER=still-TOP EM1\\
	\trans ‘Who thinks they still don’t run up to some place as they had done in the past? (Leader’s talk, spoken by Ageu at Aifa, 1998)
\end{exe}

%(6) 
\subsubsection{kaɳa EM6} 
The speaker expresses the unlikeliness of an event. 

%a.  
An event has taken place,  and the speaker had mistakenly counted on a third person (his father-in-law)  to act in a certain way.  Line (\ref{ex:eb11}) has the EM11 from Set C which marks polite regret to the listener for his agreeing to let his father-in-law do so.

\begin{exe}
	\ex
	\begin{xlist}
	\ex \label{ex:eb10}
	\gll ige-tomi=\textbf{kaŋa}-fa igei\\
	take.away-PURP=EM6-TOP IDEM\\
	\trans ‘Although it was unlikely  I would be escorted/led for that’,
	\ex \label{ex:eb11}
	\gll ukw-oto-feke=\textbf{kiŋi} a-tïfïgï-ko ige-tomi.\\
	dual-relative-ERG= EM11 EQS-IMP-PL take.away-PURP\\
	\trans ‘regrettably  I \textbf{let} our relative (parent) be our escort/ leader’. (said by Kambe at Aifa, 1980)
	\end{xlist}
\end{exe}

%b.         
Here the speaker uses EM6 to taunt an enemy to begin shooting at him. He asserts he it’s not likely he can rely on the interlocutor to bring him some arrows.  The speaker uses the \textit{ki}-applicative to change the verb ‘go’ from an  intransitive to a transitive ‘go to do X’ with the complement verb‘ bring’ .

\begin{exe}
	\ex \label{ex:eb12}
	\gll uege=\textbf{kaŋa} te-ta-tiga u-fïgi ï-ki.\\
	you=EM6 go.to-CONT-HAB 1-arrow bring-IMP\\
	\trans  'I can \textbf{hardly count on you} to always go to bring my arrows to me (=i.e., ‘shoot me’).' (said by Kudyu at Aifa, 1980)
\end{exe}

\subsection{Set B}

Information is marked as ‘external’ to the speaker’s experience or not foregrounded. There are four members of this set. As with A, there is an assumption about the interlocutor’s thoughts, wishes, or plans, but not any direct hearsay evidence. 

%(7) 
\subsubsection{\textit{nika} EM7}
The marker is a kind of strong supposition of the speaker’s regarding the interlocutor’s experience or wish.  It is often a strong marker of mourning (see example \ref{ex:eb17}).

%      a.    
A husband is anxiously looking for his wives and comes to a place where he expects to find them. He politely suggests to some other women that they have seen his wives there:

\begin{exe}
	\ex \label{ex:eb13}
	\gll aŋi=\textbf{nika} inde u-oku-ŋi-ta i-ŋi-lï e-feke-ne\\
	result=EM7 here 1-liquid.food-TRNS-CONT 3-see-PNCT 2-ERG-PL\\
	\trans ‘You \textbf{might have} seen those who make my food around here.’ (said by Tufule at Aifa, 1978)
\end{exe}

% b.  
Compounded clitics: regarding speaker’s denying asserted knowledge with interlocutor:

\begin{exe}
	\ex \label{ex:eb14}
	\gll afïtï=\textbf{nika}=wãke e-ndisï-fuŋu=\textbf{wãke}, ti-ŋiŋi-lï-i \textbf{wãke}.\\
	denial=EM7=EM1 2-daughter-resemble=EM1, 1+3-see-PNCT-COP EM1\\
	\trans ‘How could we have known if that was the daughter (\textbf{about whom you wished for}) \textbf{we saw before}?' (said by Ugaki at Aifa, 1982)
\end{exe}

\begin{exe}
	\ex \label{ex:eb15}
	\gll kuk-iñe-ti-ñï=\textbf{nika}=ale igei, nïgï-ti-feke tu-fitsu-feke.\\
	2-poison-TRNS-AN=EM7=always IDEM, QUOT–EV-ERG REF-wife-ERG.\\
	\trans '\textbf{You seem to think} this might always poison us don’t you”, that’s what he said to his wife' (said by Ugaki at Aifa, 1979)
\end{exe}

\begin{exe}
	\ex \label{ex:eb16}
	\gll kuaku iŋi-tïfïgï=\textbf{nika}-fa uege, tuwa-kua-ti.\\
	nighjar bring-IMP=EM7-PTP you, water-into-ADVï\\
	\trans ‘You’re the person the nightjars \textbf{wanted} to bring into the water, \textbf{aren’t you}’. (said by Tufule at Aifa, 1979)
\end{exe}

\begin{exe}
	\ex \label{ex:eb17}
	\gll ah u-muku-gu apuŋu-iŋo-ti=\textbf{nika} isi ki-ŋalï\\
	EXP my-son-POSS die-FUT.2-DES=EM7 3.mother utter-INCEP\\
	\trans '''The fact is my son wants to die later, \textbf{doesn’t he},”  his mother began to say.' (said by Kudyu at Aifa, 1979)
\end{exe}

%(8)  
\subsubsection{\textit{tata} EM8}
There is possibility but some uncertainty as the epistemic object involved a 3$^{rd}$ person, or a distant past source of information, and therefore the speaker could not know for certain that the event took place. However,  some probability from inferential or customary experience exists.  

\begin{exe}
	\ex \label{ex:eb18}
	\gll igifagafïtï anetu-gu uŋu a-nïgï=\textbf{tata} i-feke:\\
	{settlement name} leader-POSS house EQS-PFVS=EM8 1-ERG:\\
	\trans ‘Possibly the Igifagafiti leader has a house.’ (said by Kudyu at Aifa, 1979)
\end{exe}

%b.	
A woman recounts another's actions based on common female experience, but since the event took place in the far distant past, the speaker isn’t entirely certain of the accuracy of what she’s saying.

\begin{exe}
	\ex \label{ex:eb19}
	\gll lepene atu-ndi-li=\textbf{tata} i-feke\\
	then tap-TRNS-PNCT=EM8 3-ERG\\
	\trans ‘Then I think she might have tapped on it’ (said by Tufule at Aifa, 1979)
\end{exe}

%(9)  
\subsubsection{fïna EM9} 
With this marker the speaker expresses the idea that there is some possibility but there is no direct evidence. 

%a. 	
The following is a typical remark after the Trickster seem to have acted in his usual covert manner.

\begin{exe}
	\ex \label{ex:eb20}
	\gll Taugi=\textbf{fïna}=mbe\\
	Taugi=EM9=SS\\
	\trans ‘It might have been Taugi who did that’ (said by Tufule at Aifa, 1979)
\end{exe}

%b. 
Grandmother Quail sees someone has has pulled up all her peanuts. \textbf{\textit{fïna}} appears in this example with the mirative =\textbf{\textit{seku}}, which in this example has scope over both utterances. \textbf{\textit{fïna}} has scope over the line (\ref{ex:eb22}) utterance only.

\begin{exe}
	\ex 
	\begin{xlist}
	\ex	\label{ex:eb21}
	\gll tï-\textbf{seku=ma} egei u-etigite-gï-ki ga-tiga.\\
	RQ-MIR=EM5 ADEM 1-peanuts-POSS-INST make-HAB\\
	\trans ‘\textbf{What is this here?}  Someone’s messing with my peanuts.'
	\ex \label{ex:eb22}
	\gll eŋï oto-ni i-nïgï-ko=\textbf{fïna=seku}-fa.\\ 
	%otoninïgïko
	reason food-non-existent EX-PFV-PL=EM9=MIR-TOP\\
	\trans '\textbf{Could it be that’s} because they don’t have any food of their own?’ (said by Kambe at Aifa 1979)
	\end{xlist}
\end{exe}

%c. 
The forest monster is frightened by a man who disguises himself as an \textit{adjafi} owl in order to scare him away (this owl is a bad sign to the observer).

\begin{exe}
	\ex \label{ex:eb23}
	\gll adyafi=\textbf{fïna} its-a\\
	owl=EM9 EX-CONT\\
	\trans ‘That could be an adafi I’m seeing’. (said by Tufule at Aifa, 1979)
\end{exe}

%(10). 
\subsubsection{koh EM10}  
The speaker has no knowledge of what is being described. This morpheme is most often a particle. and is also used to begin a sentence as an expressive (example \ref{ex:eb25}). The scope of \textit{koh} covers the utterance which it follows.
  
%a. 	
A husband returns and begins to burn a pile of brush in which, unknowingly,  his wife’s lover is hiding. When the man runs away to escape the fire, the husband wonders why his wife never told him. 


\begin{exe}
	\ex \label{ex:eb24}
	\gll ukuge wende e-ki-nu \textbf{koh}-i u-feke\\
	person over.there 2-utter-non-existent EM10-COP 1-ERG\\
	\trans ‘\textbf{Why} didn’t you tell me there was someone over there? (said by Olafu at Aifa, 1979)
\end{exe}

%b. 	
Several brothers, frustrated in not having found their sister in a nearby settlement, say the following to one another:


\begin{exe}
	\ex \label{ex:eb25}
	\gll uwa=\textbf{ma} igei uk$^w$-iŋandsu i-ñïgï.\\ %migei
	what=EM IDEM dual-sister EX-TR\\
	\trans ‘\textit{What} could have happened to our sister?’
\end{exe}

\begin{exe}
	\ex \label{ex:eb26}
	\gll \textbf{koh}.la=gele-fa uk$^w$-iŋandsu its-ani\\
	EM.like.that=still-TOP dual-sister EX-FUT.1\\
	\trans '\textbf{I don’t know}, our sister might  be over there somewhere still.’ (said by Ugaki at Aifa, 1980)
\end{exe}


\section{Discussion of Sets C (Contraspective), and D (Counter-expectation)}\label{s:eb3}

\subsection{Set C}
Contraspective: in the sense there is a imaginative wishing or hoping  (sometimes, in vain) for a difference in what has been told to, or what is being observed by, the speaker. 

%(11) 
\subsubsection{\textit{kiŋi}; \textit{kiŋi-ni} (plural) EM11} 
This is used to mark regret for what has been told to the speaker. This contraspective marker is a negative nominalization of the verb \textit{ki} ‘utter’. The following example also appears in (\ref{ex:eb11}); the speaker regrets having participated in a joint venture.

\begin{exe}
	\ex \label{ex:eb27}
	\gll uk$^w$-oto-feke=\textbf{kiŋi} a-tïfïgï-ko ige-tomi.\\
	dual-relative-ERG=EM11 EQS-IMP-PL take.away-PURP\\
	\trans 'I regret I let our relative (parent) be our escort.' (said by Kambe at Aifa, 1980)
\end{exe}

%b. 
As in this example EM11 often appears hosted by the agreement expression \textit{eh he} whereby a speaker acknowledges affirmatively what the interlocutor has said but is expressing regret. (other examples include \ref{ex:eb50d}; \ref{ex:eb51b}; 25aii-iii?) %TODO probably example (\ref{ex:eb63b})--(\ref{ex:eb63c})

\begin{exe}
	\ex \label{ex:eb28}
	\gll iŋ-ke nïgifeke, ohsi-fa-ta-i ku-mugu opi-tsomi-feke figei\\
	see-I QUOT, HORT-tell-CONT-COP 1+2-son avenge-PURP=ERG IDEM\\
	\trans ‘“Look”, he said to him, “be sure to tell the others this (payment) serves to avenge our son”.’
\end{exe}

\begin{exe}
	\ex \label{ex:eb29}
	\gll e-iña u-e-ta nïgifeke.\\
	2-DAT 1-come-CONT QUOT\\
	\trans “That’s why I’ve come to you.”  
\end{exe}

\begin{exe}
	\ex \label{ex:eb30}
	\gll {eh he} \textbf{kiŋi}, {eh he}\\
	agreement EM11, agreement\\
	\trans ‘If only it weren’t so, all right’. (said by Tufule at Aifa 1979)
\end{exe}

%(12) 
\subsubsection{\textit{muk$^w$e} EM12} 
This marks the restrained or suppressed character of the utterance, an indication of the  speaker politely expressing a positive wish or expectation.  There can also be a realization on the speaker’s part that the proposition may be in vain.  This is a frequent marker in narratives; see also the conversational examples (\ref{ex:eb40b}), (\ref{ex:eb48b}), (25a,i, ii,iii?), (\ref{ex:eb70b}). %TODO probably example ( \ref{ex:eb63a})--(\ref{ex:eb63c})

\begin{exe}
	\ex \label{ex:eb31}
	\gll u-ño its-iŋa=\textbf{muk$^w$e} ukuge\\
	1-husband EX-SN=EM12 human\\
	\trans ‘If only he were human he might  be my husband.’ (said by Ugaki at Aifa, 1979)
\end{exe}

%b.
This example shows the speaker’s  use of \textbf{\textit{muk$^w$e}} to mark his restraint and modesty as a son-in-law proposing a work party to help the family. (The full conversational context appears as examples \ref{ex:eb40}--\ref{ex:eb41}).

\begin{exe}
	\ex \label{ex:eb32}
	\gll eŋï=\textbf{muk$^w$e}-tsï-fa fesoko apuGi-tsofo=\textbf{muk$^w$e}-tsï-fa ku-pehe-ne\\
	do=EM12-M-TOP (fish.name) flavor-USIN=EM12-M-TOP 1+2-ERG-PL\\
	\trans 'This way hopefully we’ll make some flavoring for our fesoko fish.'
\end{exe}

\subsection{Set D}

\emph{Counter-expectation or reevaluation of evidence} with regard to the receipt of new information.  The speaker corrects or denies their own proposition.  There are three markers.

%(13) 
\subsubsection{\textit{maki} EM13} 
The speaker reevaluates what s(he) knows.  New (introspective) information results in reevaluation of speaker’s own earlier proposition.  In (\ref{ex:eb33}): the adverb \textit{ande} is a deictic feature. To know there is a prior proposition comes from  the context of the narrative in which the speaker claims to have to travel in another direction which the interlocutor says won’t take him to the river.

\begin{exe}
	\ex \label{ex:eb33}
	\gll ande=\textbf{maki} fanguiŋga=lefa, nïgifeke.\\
	here.now=EM13 river=SEQ, QUOT\\
	\trans ‘Ok, now I see that the river is here after all”, he said to (him)’. (said by Kambe at Aifa, 1982)
\end{exe}

In (\ref{ex:eb34}), the speaker, who had thought to be a member of the Trumai group, now realizes he has been abducted as a child and says: 

\begin{exe}
	\ex \label{ex:eb34}
	\gll afïtï=\textbf{maki} Tugumai-fuŋu ku-kuge\\
	denial=EM13 Trumai-resemble 1+2-people\\
	\trans ‘not I realize we people resemble Trumai’, i.e., ‘I realize we people are are not Trumai’. (said by Ausuki at Aifa 1982)
\end{exe}

% (14)  
\subsubsection{\textit{makina} EM14} New (introspective ) information leads to acceptance of the interlocutor’s earlier proposition which was rejected: 

%a.     
A man has killed his mother-in-law but has told his wife that the woman has died from a fish on which she choked to death.  The narrator begins this part of his story with a description of what the wife sees (her first-hand evidence (\ref{ex:eb35a}--\ref{ex:eb35b})), which is followed by the woman’s surprised reaction to this evidence:  her  use of the \textbf{\textit{ah}} expressive to assert her understanding of the truth of the event, followed by her acceptance of the husband-listener’s deception,  using EM17 and the mirative \textit{ki}.

\begin{exe}
\ex \label{ex:eb35}
	\begin{xlist}
	\ex \label{ex:eb35a}
	\gll ege-te=gele is-isi-ŋa=gele\\
	PDEM-DIS=still 3-throat-inside=still\\
	\trans ‘Still there inside her throat’
	\ex \label{ex:eb35b}
	\gll uluGi akï-tïfïgï i-fu-tsï-tsïgï-fa i-feke\\
	{fish name} stuck-IMP 3-put.inside-VPE-TOP 3-ERG\\
	\trans ‘the stuck ulugi fish he had put inside her.’
	\ex \label{ex:eb35c}
	\gll \textbf{ah}, aŋ-olo=dye-tsï=\textbf{makina=ki} igei\\ %aŋolodyetsïmakinakigei
	EXP, EQS-ADV=SS-M=EM14=MIR IDEM\\
	\trans ‘The fact is I realize now that’s true,  I didn’t expect you to tell me that about her.’ (said by Kudyu at Aifa 1979)
\end{xlist}
\end{exe}

%b.  
A man uses the polite reference to his parent-in-law, \textit{uk$^w$-oto-fo-ko}. 

\begin{exe}
	\ex \label{ex:eb36}
	\gll uk$^w$-oto=fo-ko=\textbf{makina} ege, uk$w$-oto-fo-ko\\
	dual-parent-PL=EM.28 he dual-parent-PL\\
	\trans ‘Now I realize that’s not our parent, our parent.’
\end{exe}

%c. 	
A woman tells her brothers her abductor in the past was not one of their kind:

\begin{exe}
	\ex \label{ex:eb37}
	\gll ah, ukuge-fïŋï=\textbf{makina=wãke} igei=\textbf{wãke} u-iki-dyu=lefa u-feke tsa=l=ifeke, i-ño-pe-feke.\\
	EXP, 1-person-unlike=EM14=EM1   IDEM=EM1, 1-abduct-PNCT=SEQ     1-ERG   tell-always=ERG,  3-husband-SEL-ERG\\
	\trans ‘The fact is, the one who abducted me before wasn’t a person like us as I now realize’, she kept telling them about this other husband of hers’. (said by Ugaki at Aifa, 1980)
\end{exe}


%(15) 
\subsubsection{\textit{pile} EM15}  The speaker moves from acceptance to rejection of a proposition.

%a. 
A young man has agreed throughout the narrative to go to dangerous places to get things  for his brother (whose wife is his lover), but finally realizes the brother is trying to kill him. 

\begin{exe}
	\ex \label{ex:eb38}
	\gll a-faŋa-ŋo-ofo-i \textbf{wãke} u-a-nïgï=\textbf{pile} u-ikuki-ne-ta\\ %afaŋaŋofo-i           
	2-ear-NLOC-USIN-COP EM1 1-EQS-PFV=EM15 1-send.away-TRNS-CONT\\
	\trans ‘\textbf{I thought before} that because I was your co-spouse I could (safely) be sent away (\textbf{but I now understand differently})’ (said by Tufule atAifa 1979)
\end{exe}

\begin{exe}
	\ex
	\gll u-i-gu=\textbf{pile}-fa ata-ni\\
	1-ornament-POSS=EM15-TOP EQA-FUT.1\\
	\trans ‘\textbf{I mistakenly thought} my pubic ornament would be here.’ (said by Kambe at Aifa, 1979)
\end{exe}


\section{Discussion of Sets E and F}\label{s:eb4}

\textbf{Sets E} (five examples) and \textbf{F} (eight examples) concern the speaker’s marking of epistemic alignment, that is,  confirmation of a shared (or in two cases, disputed) proposition.  These sets are probably large because they are used when there is a speaker’s need to shift between a focus upon the epistemic object; the marking of a unique or shared perspective ; and the marking of the interpersonal alignment,  that is, participation with the interlocutor in an epistemic context involving a joint venture.  As is often the case, such contexts themselves change throughout  a conversation and there is often ‘disclaimed’ responsibility in which EM play a significant role.  My examples include a number of examples of dialogic contexts that illustrate the  co-construction of epistemic and evidential meaning.  Examples of markers from sets A-D appear in these longer examples. Speakers in these examples discuss issues of responsibility, denial of conjoint participation in a pragmatic context, and engagement in deception while using the affinal civility register (\citealt{Basso2007}).

\subsection{Set E}

Markers in this set concern shared information and reference the interlocutors participating in a joint venture; one negative marks refusal to share information. Conjunct/disjunct contrasts occur and degrees of knowledge are also marked.

%16.
\subsubsection{\textit{tafa} EM16} 
The speaker asserts their own (or a 1$^{st}$ person plural) alignment with the listener. The example shows both the speaker and interloctor’s use of the ‘distant future’ FUT.2, %TODO glossing 
as a reciprocal pragmatic politeness marker (see \citealt{Mendoza2016} on use of future as a politeness strategy). 

%a.. 
Cuckoo’s mother tells him to come home right away from his uncle’s settlement if the man’s  daughter he wishes to marry decides she doesn’t want him after all. Cuckoo’s answer is in Line (\ref{ex:eb39b}).

\begin{exe}
\ex \label{ex:eb39}
	\begin{xlist}
	\ex \label{ex:eb39a}
	\gll ñafe-tsï-fa iñandsu-feke e-tifu-ñe-tote, e-n-ïm-iŋo\\
	quickly-M-TOP sister-ERG 2-reject-TRNS-HYP, 2-return-INTR-FUT.2\\
	\trans ‘You’ll come back quickly if the sister rejects you, won’t you.’
	\ex Cuckoo answers his mother:\label{ex:eb39b}\\ 
	\gll s-agage-dye=\textbf{tafa} u-e-n-ïm-iŋo ama\\
	3-same-SS=EM.16 1-return-INTR-FUT.2 Mother (vocative)\\
	\trans ‘I will come back, Mother, if she is the same (as you've said).’ (said by Kudyu at Aifa, 1979)
\end{xlist}
\end{exe}

%17. 
\subsubsection{\textit{kafa} (EM17)} 
A logophoric marker, \textbf{\textit{kafa}} indicates the speaker’s presumption of a 3$^{rd}$ person’s wants,  feeling, or experience.   In this narrative section from the same story as (\ref{ex:eb36a}), four epistemic markers appear.  The speaker  affirms the 3$^{rd}$ person has accepted their proposition and asks his wife to accept a trip to get salt, which he apparently wants to do to benefit the family.  The noun ‘our parent’ used for ‘parent-in-law’ and \textbf{\textit{muk$^w$e}} EM12 are especially typical of the affinal civility register he is using, and the markers of politeness clearly enable what turns out to be his terribly deceptive speech. 

\begin{exe}
\ex \label{ex:eb40}
	\begin{xlist}
	\ex  use of ‘our parent’ as politeness marker \label{ex:eb40a}\\
	\gll aŋi=\textbf{kafa} \textbf{uk$^w$-oto-i} ŋukugu=mbo-lï ta i-feke, ta-\textbf{ki} i-feke\\ %kifeke
	result=EM17 dual-parent stay.behind=HYP-PNCT QUOT 3-ERG, QUOT-MIR 3-ERG\\
	\trans ‘“This parent of ours could have stayed behind \textbf{herself} after all”, he said to her, surprising her by saying that.’
	\ex use of \textbf{muk$^w$e} to mark a son-in-law’s restraint and modesty \label{ex:eb40b}\\
	\gll eŋï=\textbf{muk$^w$e}-tsï-fa fesoko apuGi-tsofo=\textbf{muk$^w$e}-tsï-fa ku-pehe-ne\\
	do=EM12-M-TOP (fish.name) flavor-USIN=EM12-M-TOP 1+2-ERG-PL\\
	\trans 'This way hopefully we’ll make some flavoring for our fesoko fish.'
	\ex \label{ex:eb40c}
	\gll {eh he} nïgifeke\\
	greement QUOT\\
	\trans '“All right” she said.'
\end{xlist}
\end{exe}

In line (\ref{ex:eb41a}), the speaker uses compounded markers EM22 \textbf{\textit{taka}} for suggestion of a 2p %TODO glossing?
prior agreement and \textbf{\textit{nafa}} logophorically for marking the third person;  the speaker also uses the  EM2 \textbf{\textit{tifa}} to affirm her need or wish.  
Of interest is the narrator’s final comment about how even though he is lying to his wife,  the speaker is able to convince her by using these epistemic markers that he should take his mother-in-law to the salt plants. 

\begin{exe}
\ex \label{ex:eb41}
	\begin{xlist}
	\ex \label{ex:eb41a}
	\gll uge=\textbf{taka}=\textbf{nafa} uk$^w$oto–iña=\textit{tifa} ikï-ponde-fï u-feke\\
	me=EM22=EM20 dual-parent-BEN=EM2 drying.rack-arrange-ADV 1-ERG nïgï-fi-feke.\\
	\trans “\textbf{I recall you told me (about her)} that our parent wanted me to be responsible for setting up the drying rack for her benefit ”. ‘He must have said something like that.’
	\ex \label{ex:eb41b}
	\gll eŋgu-Gi-ta i-feke\\
	deceive-TRNS-CONT 3-ERG\\
	\trans ‘He was deceiving her.’ (said by Kudyu at Aifa 1979)
\end{xlist}
\end{exe}


%18.
\subsubsection{\textit{kato} (EM18)}  
Another marker with logophoric uses, a strategy for gossipy speech as the speaker is sharing worrisome information regarding a 3$^{rd}$ person with  an interlocutor,  and denies any responsibility for what is being said. The speaker confirms what the 3$^{rd}$  person has said,  but does not always accept it as legitimate.  As the speaker aligns in this manner with the 3$^{rd}$ person outside the present speech context the shared proposition is puzzling, or even worrisome. 

%a. 
From the Kwambï,  a ritual song during which gossip about the singer is ‘thrown back’ at the original speakers:

\begin{exe}
	\ex \label{ex:eb42}
	\gll kuGife-mbe=\textbf{kato} ti-ka-gi-ti i-feke\\
	witches.dart-SE=EM18 REF-make-TRNS-DES 3-ERG\\
	\trans 
\end{exe}

%b.   
People are trying to kill jaguars. They come to a community to see if some youths can be prepared as warriors. The boys are tested by killing a tapir:  In Line (\ref{ex:eb43a}) the speaker uses the third person logophiric \textbf{\textit{kafa}} because the childen’s actions confirms their proposal of a joint venture.  In Line \ref{ex:eb43b}, there is use of \textbf{\textit{kato}}. Here the proposal that the children might be able to kill the jaguars is considered possible by the speaker but still somewhat weak and worrisome, which is given further context by the narrator’s explanatory material in Line \ref{ex:eb43c}.

\begin{exe}
\ex \label{ex:eb43}
	\begin{xlist}
	\ex \label{ex:eb43a}
	\gll agetsi-tsi=mbembege=dya=\textbf{kafa} ule-tsaŋe\\
	one-M=PE=DS=EM17 AFR-DEO\\
	\trans ‘\textbf{They seem} to have wanted to do that together to it (the tapir),  what  we had them do.’
	\ex \label{ex:eb43b}
	\gll ata-dye=\textbf{kato} ku-mugu-ko-feke ukw-opi-dyï-ko-iŋo\\
	EQA-SS=EM18 1+2-son PL-ERG 1+2-avenge-PNCT- PL-FUT.2\\
	\trans ‘”I’m not sure, but since they’ve done it that way,  our children \textbf{must want to} avenge us later”.’
	\ex \label{ex:eb43c}
	\gll etsï-ŋi-nda-ko=mbedya-fa\\
	fright-INTR-CONT-PL=SE.DS-TOP\\
	\trans ‘They were still frightened by what those others (jaguars) kept doing.’ (said by Saŋafa  at Aifa)
\end{xlist}
\end{exe}

%19. 
\subsubsection{\textit{tiki} EM19} This marks the speaker’s refusal to accept a role in an endeavor described by the interlocutor or by third persons.  \textbf{\textit{tiki}} is also a mirative form that emphasizes the speaker’s sudden surprise concerning the contrastive understanding or disagreement.  The form may be derived from the negative mirative suffix –\textbf{\textit{ki}} (see also example \ref{ex:eb49f}--\ref{ex:eb49g} with \textit{tifa}). 

%a. 
A woman has been abducted by powerful beings and left in her hammock tied to a tree in the wilderness.  When she wakes up, she says:

\begin{exe}
\ex \label{ex:eb44}
	\gll una=\textbf{tiki} egei u-e-tïfïgï\\
	Q=EM19 ADEM 1-come.to-IMP\\
	\trans  ‘How in the world did I (or ? could I have wanted to’) come here like this?’ (said by Tufule at Aifa, 1979)
\end{exe}

%b. 
A warrior from a cannibal group cannot understand why his future wife says others fear he has come to kill them all, because he’s only come to marry her. (see earlier part of this conversation in (\ref{ex:eb71}).

\begin{exe}
\ex \label{ex:eb45}
	\begin{xlist}
	\ex \label{ex:eb45a}
	\gll una-male ago te-ta figei nïgifeke\\
	Q-so.many these.people go.away-CONT ADEM QUOT\\
	\trans ‘Why are so many of these people here going away like this’, he asked her.
	\ex \label{ex:eb45b}
	\gll ñ-eŋe-tu-nda-ko tsale igei e-feke\\
	DE-fear-TRNS-CONT-PL 3-so many IDEM 2-ERG\\
	\trans ‘They’re frightened of what you’re like’.
	\ex\label{ex:eb45c}
	\gll tuelï-ko faŋa-mi-ta igei efeke.\\
	kill-PNCT-PL worry-adversative-CONT IDEM 2-ERG\\
	\trans ‘They’re worried that you will kill them all like this.’  
	\ex \label{ex:eb45d}
	\gll ta-\textbf{tiki} Ø-e-lï-ko-iña u-e-nalï\\
	RQ-EM19 3-kill-PNCT-DAT 1-come.to-MAL\\
	\trans ‘Why should anyone say I’ve come here to kill you all?’
	\ex \label{ex:eb45e}
	\gll awu-nda-fïŋï=mbedye tsa=lefa.\\
	like-CONT-resemble=SE-SS EX=SEQ\\
	\trans ‘What they’re saying is like a lie’. 
	\ex \label{ex:eb45f}
	\gll e-lï-ko-iña-la tale igei u-e-tïfïgï, nïgifeke\\
	kill-PNCT-PL-DAT-NEG NEG.always IDEM 1-come-PFV, QUOT\\
	\trans ‘I never came like this to kill you all’
	\ex \label{ex:eb45g}
	\gll fïgei-pe apokine-nïgï=mbe=dye tsa i-feke ule-fa\\
	arrow=SEL put.down-PFV=SE-SS EX 3-ERG AFR-TOP\\
	\trans ‘He had already put down his arrows because of what she was saying.’
	\ex \label{ex:eb45h}
	\gll iñalï nïgifeke\\
	no QUOT\\
	\trans ‘Not so’, he said. (said by Apihũ at Aifa, 1967)
\end{xlist}
\end{exe}

%20. 
\subsubsection{\textit{nafa} (EM20)} 
With this marker, a 1$^{st}$ person responds to the interlocutor regarding  a 3$^{rd}$ person animate non-human agent or subject participant.   \textbf{\textit{nafa}} is used in regard to the speaker  assuming the interloctor has customary knowledge of the consequences of the context. 

\begin{exe}
	\ex \label{ex:eb46}
	\gll u-ifi-fo-lu=\textbf{nafa} e-feke sike-feke-fa e-fife-po-lï\\
	1-touch-HYP-PNCT=EM20 2-ERG tocandira.ant-ERG-TOP 2-bite-HYP-PNCT\\
	\trans ‘You must know if you were to touch me  the tocandira ant would sting you.’ (said by Tufule at Aifa, 1979)
\end{exe}
 
% b. 
While introducing the character Cuckoo, the storyteller confirms that I already know what he’s talking about, as he was keenly aware I had already worked with others identifying Kalapalo bird names.

\begin{exe}
	\ex \label{ex:eb47}
	\gll fitsagu, tu-fu-ti-sï=\textbf{nafa} e-feke\\
	cuckoo, REF-know-TRNS-PNCT=EM20 2-ERG\\
	\trans ‘Cuckoo, you must already know yourself what that is.’ (said by Kudyu at Aifa, 1982)
\end{exe}

%c. 
The following example describes people escaping a massacre who are giving out dead hummingbirds as food to their fellow travellers. What occurs as a consequence is a case of oracular interpretation insofar as responsibility for a problematic decision is displaced deictically away from the discourse of the 2$^{nd}$ person human locutors and onto the 3$^{rd}$ person non-human epistemic object.  In Line (\ref{ex:eb48b}), there is a use of \textbf{\textit{muk$^w$e}}  (EM12)as ‘hoping in vain’. In Line (\ref{ex:eb48c}), there is a use of \textbf{\textit{maki}} (EM13) marking change of opinion after receipt of new information (see also line iv). Line (\ref{ex:eb48d}) shows EM13 suffixed by the negator –\textit{la} (NEG) %TODO glossing 
as are final elements of the other clauses.  

\begin{exe}
\ex \label{ex:eb48}
	\begin{xlist}
	\ex \label{ex:eb48a}
	\gll lepe, ohsi ku-ñi-kondï-ŋi\\
	next, HORT 1+2-DE-give.out.NN\\
	\trans ‘Then  “let’s see whether or not we can share them (lit., our not sharing)”.’
	\ex \label{ex:eb48b}
	\gll tï-kon-di Ø-feke-ni, pok, pok, pok, katote itau=\textbf{muk$^w$e}-feke kugiti\\
	REF-give.out-ADV 3-ERG-PL, (put down sound ), all woman=EM12-ERG everyone\\
	\trans ‘They gave them out, the women hoped to give them out to everyone.’
	\ex \label{ex:eb48c}
	\gll he=dye=\textbf{maki}.\\
	yes=SS=EM13\\
	\trans ‘Yes, that’s \textbf{not what I/we expected}.’
	\ex \label{ex:eb48d}
	\gll afïtï=dye=\textbf{maki}-la aŋikogo-feke kukw-e-lï-ko-la, kukwe-lï-ko-iŋo-la, afïtï\\
	denial-SS-EM13-NEG fierce.people-ERG 1+2-kill-PNCT-PL-NEG, 1+2-kill-PNCT-PL-FUT.2-NEG, denial.\\
	\trans “Since that wasn’t \textbf{what I/we expected}, the fierce people never  kill us all, will never  kill us all, not so.”
	\ex \label{ex:eb48e}
	\gll i-ŋa-po-lï=\textbf{nafa} its-a-ini iñopi-fo-lï-ko=lefa ule-tseŋugu-i\\
	3-left-over-HYP-PNCT=EM20 EX-CONT-PL go.back-HYP-PNCT-KO=SEQ AFR-interrupt-COP\\
	\trans ‘Had \textbf{they found} there was not enough (\textbf{of them}) for everyone,  they would have stopped that (i.e.travelling) and gone back for that reason.’ (said by Ausuki at Aifa 1982)
\end{xlist}
\end{exe}

%d.  
In this example, an entire conversation is presented to show how EM19 emerges at the end of presentation of knowledge by the interlocutors.  This is when the speaker expresses surprise and also questions knowledge. This utterance is constructed with a rhetorical question that emphasizes the speaker’s sudden and surprising realization of the truth of what is being told him: Note in line (\ref{ex:eb49d}). there is a compounded hearsay EV and EM20. 

\begin{exe}
\ex \label{ex:eb49}
	\begin{xlist}
	\ex \label{ex:eb49a}
	\gll apiči, nïgifeke\\
	Grandfather, QUOT\\
	\trans ‘Grandfather’, they said to him.’
	\ex \label{ex:eb49b}
	\gll tisuge-ake egei efigï ake.\\
	1+3-COM ADEM 2-grandson+COM\\
	\trans ‘Together  he and I, we’re your grandsons.”
	\ex \label{ex:eb49c}
	\gll tï-\textbf{ma}=ale-i u-figï-i\\
	RQ-EM4=always-COP 1-grandson-COP\\
	\trans ‘How can it be there have always been grandsons of mine?’
	\ex \label{ex:eb49d}
	\gll tisuge-\textbf{ti=nafa} egei.\\
	1+3-EV-EM20 ADEM\\
	\trans ‘That’s what \textbf{we know as we’ve been told} that about you.’
	\ex \label{ex:eb49e}
	\gll e-figï=\textbf{nafa} egei itau-kuegï muku-gu-pe,\\
	2-grandson=EM20 ADEM woman-AUG son-POSS-SEL,\\
	\trans ‘\textbf{You now know} Monstrous Woman had a son, that’s him over here.’
	\ex \label{ex:eb49f}
	\gll uge=\textbf{tifa} egei ñafïgï muku-gu-pe.\\
	me=EM2 ADEM (name) son-POSS-SEL\\
	\trans ‘\textbf{I’ve learned} I am one of ñafïgï’s sons’.
	\ex \label{ex:eb49g}
	\gll {eh he!} nïgifeke ah ande-ŋu=\textbf{tiki} u-fi-dyau a-nïgï\\
	agreement QUOT EXP here/now-DIM=EM19 1-grandchild-PL EQS-PFV\\
	\trans ‘All right!’ he answered. ‘The fact is, \textbf{I had no reason to think} you little ones here were my grandchildren!’ (said by Kambe at Aifa 1980)
\end{xlist}
\end{exe}


\subsection{Set F}

This set marks differences between the speaker and listener regarding a speaker’s agreement to participate in the proposal of a shared context for information.   The speaker appeals to a listener with a marked grade of confidence,  referencing the conjoint situation.  There is a contrast between these and EM19, which marks the speaker’s denial of any conjoint agreement.

%21 
\subsubsection{\textit{aka} EM21} The speaker substantiates the interlocutor’s  proposition and information is shared by the speaker with the listener.  The speaker declares there is existing, positive alignment between herself (often as a member of a non-inclusive plural group) and the listener.  In example (\ref{ex:eb50a}), the speaker uses \textbf{\textit{nika}} (EM7 Set B) to align with the interloctor’s experience.
In Line (\label{ex:eb50d}), he uses \textit{eh he \textbf{kiŋi}}, regretful agreement.

%a. 
Based on his  own experience, Cuckoo judges his wife’s family’s eyes are constantly bothered by the smoke in their house. 

\begin{exe}
\ex \label{ex:eb50}
	\begin{xlist}
	\ex \label{ex:eb50a}
	\gll ti-fati, uma=ale igei igea=\textbf{nika}=ale figei e-iñalï-ko\\
	REF-ask-ADV, Q=always IDEM manner=EM7=aways ADEM 2-MAL-PL\\
	\trans ‘Asking,  “Aren’t you all always bothered this way like I am?”’
	\ex \label{ex:eb50b}
	\gll \textbf{eh}. ŋi-ke-fa tisuge...\\
	yes, see-I-TOP 1+3\\
	\trans ‘Yes, look at us.’
	\ex \label{ex:eb50c}
	\gll igei-fuŋu=\textbf{aka} tisuge, igei-fuŋu\\
	IDEM-resemble=EM21 1+3, IDEM-resemble\\
	\trans ‘As you see this is what we are like, like this’. 
	\ex \label{ex:eb50d}
	\gll {eh he} \textbf{kiŋi}\\
	assent EM11\\
	\trans ‘Yes, but if only it weren’t so.’ (said by Kudyu at Aifa, 1982)
\end{xlist}
\end{exe}

%b.  
The Dead are invited to return to the living but decline. In Line (\ref{ex:eb51a}), the  use of \textbf{\textit{nifa}}, EM 25, marks that the (quoted) speaker shares a conjoint decision  with the listener.

\begin{exe}
\ex \label{ex:eb51}
	\begin{xlist}
	\ex \label{ex:eb51a}
	\gll tis-ogo-pi-tsa=\textbf{nifa} aña ki-lï\\
	1+3-go.back-INTR-CONT=EM25 Dead utter-PNCT\\
	\trans “All of us have agreed to go back”, the Dead said.
	\ex \label{ex:eb51b}
	\gll tits-e-lï=\textbf{aka} ake-ts-igei=lefa\\
	1+3-go-PNCT=EM21 SD-EX-IDEM=SEQ\\
	\trans “And so, as you see, we people must take leave of you.”’ (said by Tufule at Aifa, 1979)
\end{xlist}
\end{exe}

%22. 
\subsubsection{\textit{taka} EM22} 
This is a weak or uncertain conjunct marker. With this marker,  the speaker proposes to the listener (s)he will probably agree with the proposition but speaks somewhat uncertainly.  In my examples, =\textbf{\textit{taka}} is seen with somewhat hesitant, rather than assertive speech, as there is no direct evidence the listener can use to assert knowledge or accept the proposition.

\begin{exe}
	\ex \label{ex:eb52}
	\gll aŋi \textbf{muk$^w$e=taka} itau=go-pïŋï fogi-tsïgï u-feke\\
	exist EM12=EM22 woman-PAU-DEF find-IPE 1-ERG\\
	\trans ‘As I hoped there were a few deficient women I found by chance.’ (said by Kambe at Aifa 1979)
\end{exe}

%b.
The following example shows \textbf{\textit{taka}} hosted by an “I told you so” quotative(bracketed). The quotation, which precedes the quotative, is included in the scope of the EM =\textbf{\textit{taka}}.

\begin{exe}
	\ex \label{ex:eb53}
	\gll uk$^w$-aŋi-fofo [u-k$^w$i-ta=\textbf{taka}] egei.\\
	dual-exist-IM 1-utter-CONT=EM22 IDEM\\
	\trans ‘“Let's wait a while”.  I suppose you \textbf{might remember} I said that.’ (said by Nikumalu at Aifa 1979)
\end{exe}

%c.  
Use of \textbf{\textit{taka}} with a Q word. The speaker asks a listener to help him remember someone’s name:

\begin{exe}
	\ex \label{ex:eb54}
	\gll uwa=\textbf{taka} i-ñandsu ititï, Kamisu?\\
	What=EM22 3-sister name, (name)\\
	\trans ‘What did I say \textbf{about} the sister’s name?  Kamisu?’ (said by Ugaki at Aifa, 1979)
\end{exe}

%d.   
What follows is a conversational example regarding responsibility,  with examples of several different logophoric markers.  After travelling from place to place without finding her, the relatives searching for a woman have arrived in a community where some warriors (during the time of her abduction) had seen people on the river.  They have evidence but never knew how to interpret it until they are told several years later about this woman who has been abducted.  Their interlocutor uses \textbf{\textit{muk$^w$e}} as a politeness strategy.  In Line (\ref{ex:eb55b}), \textbf{\textit{kafa}} marks the speaker’s weak presumption of the third person’s experience.  In Line (\ref{ex:eb55e}), \textbf{\textit{nika}} is the second person logophoric marker emphasizing the speech of the visitor and \textbf{\textit{wãke}}  marks the speaker’s first hand, distant past experience used to assert his knowledge to the listener.What has happened is the warriors realize they didn’t have any way of knowing the people they saw on the river were the woman and her abductor.  If so,  they might have been able to release her. The narrator describes the following conversation: 

\begin{exe}
\ex \label{ex:eb55}
	\begin{xlist}
	\ex \label{ex:eb55a}
	\gll tseta-\textbf{ŋapa}-fa i-dye-Gi-nïgï i-feke\\
	same.place-probably-TOP 3-ask.about-CAUS-PFV 3-ERG\\
	\trans ‘Most likely they asked about her there.’ (narrator’s line)
	\ex \label{ex:eb55b}
	\gll aŋi fogi=kafa inde iñandsu e-tïfïgï inde=mukwe u-limo, nïgifeke\\
	result search.for-EM17 here  sister come-IMP here=EM12 1-children, QUOT\\
	\trans ‘“Is there a chance someone who I’m looking for may be here, I hope the sister came here my sons ”’ he said about her.’
	\ex \label{ex:eb55c}
	\gll ñalï=\textbf{ma}-e-tsaŋe e-indisï e-nïmi, ta i-feke-ni.\\
	NEG-EM4-2-DEO 2-daughter come-CONS, tell 3-ERG-PL\\ 
	\trans “We \textbf{don’t think} what you’re wanting, there’s any reason your daughter came here after all”, they answered.
	\ex \label{ex:eb55d}
	\gll ule-ŋugi=\textbf{taka}=ale egei-i unago-fa, tafaku oto-mo-fa,\\
	AFR-interrupt=EM22=uninterrupted ADEM-COP some.others-TOP, bow master-PL-TOP\\
	\trans ‘Before that (his constant talking about her) was suddenly interrupted by some others, bow masters
	\ex \label{ex:eb55e}
	\gll afïtï=\textbf{nika} \textbf{wãke}, endisï-fuŋu \textbf{wãke}, ti-ŋ-iŋi-lï-i \textbf{wãke}\\
	contrast=EM7 EM1 daughter-resemble EM1, 1+3-DE-see-PNCT-COP EM1\\
	\trans ‘How could we have known if that was the daughter (about whom you spoke) we saw before?” (said by Ugaki (Aifa, 1982)
\end{xlist}
\end{exe}

%23. 
\subsubsection{kalaka EM23} 
EM23 marks the speaker’s impressions of a vague 3$^{rd}$ person reflexivity.  The impressons are positive although the speaker may have no direct evidence regarding the 3$^{rd}$ person’s thoughts or even the identity of the person.  This contrasts with \textbf{\textit{kato}} (EM18), which marks worrisome impressions of a hearsay or gossipy nature on the part of the speaker.  Irrealis features in the examples contribute in several ways to the idea that \textbf{\textit{kalaka}}  marks an uncertainty ( because it introduces the clause marked with \textbf{\textit{koh}} EM10),  and even a polite proposition (because the clause is used with the -\textit{iŋo} FUT.2 distant future). %TODO glossing

%a.    
A woman suggests to her sisters that a new fruit might be used as a drink if the Trickster invents it. The use of the irrealis distant future FUT.2 contributes to this weak proposition. %TODO glossing

\begin{exe}
	\ex \label{ex:eb56}
	\gll aŋi fogi=\textbf{kalaka} tï-iñambe=nïm-iŋo-\textbf{koh} ta-\text{ti} i-feke\\
	result find=EM23 REF-drink-INTR-FUT.2=EM10 QUOT-EV 3-ERG\\
	\trans “ He could find \textit{in itself}  a kind of drink for us sometime”, she kept saying about it, they say.’ (said by Tufule at Aifa 1979)
\end{exe}

%b. 
From a conversation between a Kalapalo speaker and the present author about usage:

\begin{exe}
\ex \label{ex:eb57}
	\begin{xlist}
	\ex 
	\gll ege muku-gu ipo-pïgï=\textbf{kalaka} egei-i\\
	PDEM son-POSS pierce-PFV=EM23 ADEM-COP\\
	\trans ‘She might have wanted someone to pierce her son (i.e. pierce her son’s ears). ’
	\ex 
	\gll \textbf{ah}-la-tsï, ege muku-gu i-po-pïgï its-a egei.\\
	EXP-like.that-M, 3PDEM son-POSS 3-pierce-PFV EX-CONT ADEM\\
	\trans ‘The fact is that’s right, the fact is her son’s ears have been pierced.’
\end{xlist}
\end{exe}

%c. 
In this example,  the speaker suggests his mother-in-law (whom he can’t address directly)  might have an opinion different from his own.  In Line (\ref{ex:eb58b}) he uses \textbf{\textit{aka}} marking information shared with his wife. 

\begin{exe}
\ex \label{ex:eb58}
	\begin{xlist}
	\ex \label{ex:eb58a}
	\gll \textbf{ah} ñuku nïgifeke \textbf{ai} efitsu ki-lï\\
	EXP (name) QUOT EXP wife utter-PNCT\\
	\trans ‘The fact is ñuku’, he said to her.’What?’, his wife answered.’
	\ex \label{ex:eb58b}
	\gll ta-iku=nile ukwi-ta-ku=\textbf{aka} igei\\
	RQ-fully=wrong dual-CONT-fully=EM21 IDEM\\
	\trans ‘Why do we have to speak so mistakenly \textbf{to each other} about this ?”
	%Because of affinal avoidance, the speaker has no direct evidence of his mother-in-law’s wishes and so uses \textbf{kalaka} in the next line.
	\ex \label{ex:eb58c}
	\gll aŋi=\textbf{kalaka} ukw-oto iŋu-kugu-mbo-lï\\
	result=EM23 dual-parent join-fully-INTR-PNCT\\
	\trans ‘I’m not sure but maybe our parent has decided to come along \textbf{by herself}. ’ (said by Kudyu at Aifa, 1979)  
\end{xlist}
\end{exe}

%24. 
\subsubsection{\textit{nipa} EM24}  
Marks a contradiction  involving an interlocutor’s or 3$^{rd}$ person’s role treated as different from what has been experienced, said, or seen. \textbf{\textit{nipa}}  is often used in examples that reference the distant past,  but where the speaker has some historical evidence of positive impressions of 3$^{rd}$ persons.  For example in (\ref{ex:eb59}) the narrator’s  use of \textbf{\textit{nipa}} references impressions of temporally distant others, no longer present but mentioned in historical narratives. 

\begin{exe}
	\ex \label{ex:eb59}
	\gll teh \textbf{ekugu}=mbe=\textbf{nipa} u-kuge\\
	nice fully=SE=EM24 1-people\\
	\trans ‘But my (ancestor)  was  \textbf{thought} very beautful when that happened (i.e. when he painted himself) (=when the other people saw them.)’‘ (Kofoño to EB at Aifa, 1979)	
\end{exe}

%b.  
Here the complement verb, ‘see, think’ in imperative mood (\textit{iŋ-ge}), is used with the main  verb for ‘kill’.  A warrior is addressing his relative, a person with little experience:

\begin{exe}
	\ex \label{ex:eb60}
	\gll iŋ-ge elu=mbe-tsï=\textbf{nipa} a-pi-ga i-feke-ne\\
	look/think-I kill=SE-M=EM24 2-club-CONT 3-ERG-PL\\
	\trans ‘Think carefully, don’t you realize (\textbf{if you did that}) they would club us to death?’
\end{exe}

%c.   
In the context of a dispute about traveling, foreign visitors keep insisting upon going in a certain direction, even though others repeatedly show them (using arm motion)  the direction to travel so as to avoid enemies. Possibly \textbf{\textit{nipa}} is used as the participants are foreigners with no knowledge of the local geography. 

\begin{exe}
\ex \label{ex:eb61}
	\begin{xlist}
	\ex 
	\gll la! e-te-ke efu-ta=\textbf{nipa} e-ge-tomi Atatsinu-kai\\
	{that way} 2-go-I canoe-in=EM24 2-travel-PURP (name)-by.means.of\\
	\trans That’s the way you should go \textbf{different from what you said}, by canoe, on the Atatsinu.
	\ex 
	\gll afïtï, la u-te-ta-ni\\
	denial, distant.place 1-go-CONT-FUT.1\\
	\trans ‘I won’t, I’m going that (other) way.’
	\ex
	\gll i-tagiñu-pe kugu, i-tagiñu-pe male!\\
	3-speech-SEL fully, 3-speech-SEL too.much\\
	\trans ‘So much of their speech,  too much of that speech of theirs!'
	\ex
	\gll la=\textbf{nipa} e-te-ke, Atatsinu-kai=\textbf{nipa} e-te-ke, la.\\
	distant.place=EM24 2-go-I (name)-DEST=EM24 2-go-I, distant.place\\
	\trans ‘That’s the way you should go, \textbf{different from what you said},  you should go on the Atatsinu (creek),   way over that way (even though you don’t want to).’ (said by Kambe at Aifa 1982)
\end{xlist}
\end{exe}

%25. 
\subsubsection{\textit{nifa} EM25} 
The speaker invites interlocutor into a context to share knowledge, or asserts a conjoint decision regarding 2$^nd$ person or 3$^rd$ person  (see also example (\ref{ex:eb51a}).

\begin{exe}
	\ex \label{ex:eb62}
	\gll ukwatsa-ke-\textbf{nifa} ku-nitsu-na\\ 
	dual-run-I-EM25 1+2-granmother-ALL\\
	\trans ‘We really should hurry over to that grandmother of ours.’
\end{exe}

%b. 
The following example is a discussion of the need for a conjoint decision despite an oracular message to the contrary. In line (\ref{ex:eb63b}) the speaker uses \textbf{\textit{kiŋi}} (contraspective negation); and \textbf{\textit{aka}} (confirmation of 2person).  A father is asked to help prepare his young son  to be a warrior who will kill ravaging jaguars.  Unlike the earlier examples (\ref{ex:eb46})--(\label{ex:eb49}) in which the oracular message is accepted, in this case it is at first rejected but as the need to act is far more important  \textbf{\textit{nifa}} is used (Line \ref{ex:eb63c}) together with the politeness  \textbf{\textit{muk$^w$e}} ‘hope in vain’ as the speaker, a leader,  is requesting something very special of the interlocutor that will benefit the community as a whole. 

\begin{exe}
\ex \label{ex:eb63}
	\begin{xlist}
	\ex \label{ex:eb63a}
	\gll taloki=\textbf{muk$^w$e} figei eŋï=\textbf{muk$^w$e}-fa e-mugu tïi-lï ku-pehe nïgifeke.\\
	useless=EM12 ADEM do=EM12-TOP 2-son make-PNCT 1+2-ERG QUOT\\
	\trans ‘”It may be useless but even so we should try and make your son”, they said.’
	\ex \label{ex:eb63b}
	\gll {eh he} \textbf{kiŋ-ale}, afïtï-ku=\textbf{aka} u-wituŋu=mbe-su ifo-fïŋï egitse egei.\\
	agree-EM11-always, denial-INT=EM21 1-dream=SE-PEJ image-resemble unable PDEM\\
	\trans ‘If only it wasn’t so.  I want you to know I’ve never \textbf{had} such a dream image \textbf{as that}, to my detriment.’
	\ex \label{ex:eb63c}
	\gll taloki=\textbf{muk$^w$e=nifa} kuk-opiso-ko-omi-ŋo=\textbf{muk$^w$e}-fa ku-mugu-ko tu-itu-e.\\
	useless=EM12=EM25 1+2-avenge-PL-PURP-FUT.2=EM12-TOP 1+2-son-PL REF-make-I\\
	\trans ‘It may be useless \textbf{but even so} in order to try and have someone avenge us we need to make our son.’
	\ex \label{ex:eb63d}
	\gll {\textbf{eh he}} nïgifeke.\\
	agree QUOT\\
	\trans ‘All right’, he answered. (said by Kambe at Aifa, 1979)
\end{xlist}
\end{exe}

%26.  papa EM26. This marker is used when the speaker agrees the interlocutor should cooperate by joining in a proposed activity.  Use of papa seems also to be a politeness strategy in speech about affinal relatives,   as seen in both .examples   
%a. The following conjoint example involves a mother agreeing to participate in her dauughter’s  new relationship of marriage, as seen by the pair’s reciprocal use of the affinal civility register in which special politeness kin terms (daughter referring to her husband as ‘your nephew’ and her mother referring to her son-in-law as ‘your younger brother’  are used.  

\begin{exe}
\ex \label{ex:eb64}
	\begin{xlist}
	\ex 
	\gll {\textbf{u: um}}, ama nïgifeke, ama.\\
	EXP Mother QUOT Mother\\
	\trans “Mother”, she said, “Mother”,
	\ex 
	\gll ande a-fatuwï. ande a-fatuwï\\
	here/now 2-nephew, here/now 2-nephew (i.e., daughter’s husband)\\
	\trans “I’m thinking your nephew’s here now, I’m thinking your nephew’s here now.”
	\ex
	\gll {eh he} nïgifeke.\\
	agreement QUOT\\
	\trans “All right”, she answered
	\ex
	\gll ah, e-fisï-tomi=\textbf{papa} ege-na e-iña..........\\
	EXP, 2-younger.brother-PURP=EM.26 3rdp-ALL 2-BEN\\
	\trans ‘The fact is,  if you want him to come  be your  younger brother (i.e. new husband) on your behalf,  that’s all right with me.....” (Afanda told by Ugaki at Aifa 1979)
\end{xlist}
\end{exe}

%b. 
In an imperative construction, the  speaker uses papa to agree with his wife:

\begin{exe}
	\ex \label{ex:eb65}
	\gll ege-ke=\textbf{papa} kuigiku nïgïfeke, ukw-oto-iña, ukw-oto-iña.\\
	2-take away-I=EM26 manioc.soup QUOT dual-parent-BEN, dual-parent-BEN\\
	\trans “I agree you should take this soup with you”, he said to her, “for our parent, for our parent”. (said by Ugaki at Aifa 1979)
\end{exe}

%27. 
\subsubsection{\textit{apa} EM27} 
This EM appears to be informal in comparison with EM 26.  The speaker wants a 2$^{nd}$ or 3$^{rd}$ person to cooperate as asked. (1 cooperates with 2 as in (\ref{ex:eb66a}); 1 asks 2 to cooperate as in (\ref{ex:eb67}--\label{ex:eb68})).  

\begin{exe}
\ex \label{ex:eb66}
	\begin{xlist}
	\ex \label{ex:eb66a}
	\gll u-limo-wï-tsïpïgï-fa fu-mi-kege u-wetiko-gu-ki e-te-tomi aikaku-na, uguka-ki.\\
	1-child-father-unending-TOP send-TRNS-I 1-belt-POSS-INST 2-go=PURP (name)-ALL, shell.ornaments-INST\\
	\trans ‘Send the father never having my children (politeness locution)  to get my belt(s) from the  Aikaku, to get shell ornaments (s).” 
	\ex \label{ex:eb66b}
	\gll {eh he} nïgifeke\\
	agreement QUOT\\
	\trans ‘All right” she said to him.
	\ex \label{ex:eb66c}
	\gll u-ki-ta-ni=\textbf{apa}=fofo i-feke i-fitsu ki-lï\\
	I-utter-CONT-FUT.1=EM27=IMM 3-ERG 3-wife utter-PNCT\\
	\trans “I’ll speak to him right away \textbf{just as you wish}, his wife said. (said by Tufule at Aifa 1979) 
\end{xlist}
\end{exe}

\begin{exe}
\ex \label{ex:eb67}
	\begin{xlist}
	\ex 
	\gll agetsi-ŋo-i-tsï=\textbf{apa} ku-te-ga-ni-ni,\\
	one-NLOC-COP-M=EM27 1+2-go-CONT-PL-FUT.1\\
	\trans ‘I \textbf{want} us to all go together soon,
	\ex 
	\gll kule-mi-la=ale=keñi e-mugu-ko undu-pesi-kïgï i-feke-ni\\
	take.care-TRNS-NEG-always=beware 2-son-PL attack-ugly-PFV 3-ERG-PL\\
	\trans ‘as we can’t ever stop watching out in case they make an attack on your son.’ (said by Madyuta at Tangugu, 1979)
\end{xlist}
\end{exe}

\begin{exe}
	\ex \label{ex:eb68}
	\gll \textbf{ah}, e-ŋi-ke=\textbf{apa} ande la-iña its-apï-gï-ko its-a, tï-fitseŋe-ki-ñï akago\\
	EXP, 2-see-I=EM27 here/now-NEG-DAT 3-footprint-POSS-PL EX-CONT, REF-stink-INTR-SN  those. people\\
	\trans ‘The fact is, you can see their footprints here just as I did,  those people over there are the  Stinking People whom I want you to see’. (said by Madyuta at Tangugu, 1979)
\end{exe}

\begin{exe}
	\ex \label{ex:eb69}
	\gll i-ŋi-gote-fa ku-pehe safake-fi-tsïgï-tsï=apa e-ta....,\\
	3-bring-CONC-PTP 1+2-ERG 3-body.trunk-crush-IPE-M-EM27 come-CONT\\
	\trans “If we want to bring (some of it), ‘come on let’s go get ‘it’s‘crushed body’ like you did....”. (said by Kudyu at Aifa 1979)	
\end{exe}

%28. 
\subsubsection{\textit{tima} EM28}  
A contrastive (1≠2)used when the speaker wishes to assert a strong disclaimer of responsibility, particularly a denial of fault after being blamed by an interlocutor. The retorical question prefix \textit{tï} contributes to the sense of angry negation. Context: The story concerns a  maned wolf father who is looking for his son‘s poisoner,  going from one plant to another: 

\begin{exe}
\ex \label{ex:eb70}
	\begin{xlist}
	\ex \label{ex:eb70a}
	\gll tï-tomi=\textbf{tima} e-mugu igi-fes-iñalï u-feke\\
	RQ-PURP=EM28 2-son head-ugly-MAL 1-ERG\\
	\trans ‘Why should I be \textbf{blamed} for bothering your son?’
	\ex \label{ex:eb70b}
	\gll u-ikeu-te-la=\textbf{muk$^w$e-ti} igitse e-mugu ё-ta u-fanu-tsoke-tiGi\\
	1-be.angry-TRNS-NEG=EM12-DES unable 2-son come-CONT 1-?-chew-ADV\\
	\trans ‘I never tried to get angry when your son came to chew open my fruit.’
	\ex \label{ex:eb70c}
	\gll tï-tomi=\textbf{tima} u-ikeu-ŋalï i=feke\\
	RQ-PURP=EM28 1-be.angry-MAL 3-ERG\\
	\trans ‘\textbf{Why should I} be \textbf{blamed} for getting angry with him?’ (said by Kudyu at Aifa 1979)
\end{xlist}
\end{exe}

%b. 
Context: The warrior comes to marry a young woman, but as he is always clutching his bundle of arrows to his chest, her mother is very frightened.  

%TODO alignment of first example
\begin{exe}
\ex \label{ex:eb71}
	\begin{xlist}
	\ex 
	\gll iŋke-fa nïgifeke, ukw-apïŋï fegei\\
	look-I-TOP QUOT dual-die-PFV ADEM\\
	\trans ‘Think of it, she said, ‘Someone like that will make us die’.
	\ex 
	\gll ah tafako oto ale elei, kukw-e-luiŋo fegei i-feke.\\
	EXP bow master CUM PDEM 1+2-kill-FUT.2 ADEM 3-ERG\\
	\trans ‘Believe me, they’re all bow masters over there, as someone like that he’s going to kill us.’
	\ex
	\gll {eh he} ta-\textbf{tima} afïtï\\
	agreement RQ-EM28 contrastive\\
	\trans ‘How can I be blamed, that’s wrong.’
	\ex
	\gll ta-te-ŋalï-ko u-feke, \textbf{u.um}, ta-te-ŋalï-ko u-feke.\\
	NEG-go-MAL-PL 1-ERG EXP NEG-go-MAL-PNCT-PL 1-ERG\\
 	\trans ‘\textbf{Why should I} kill them?  I’m trying to figure out why they think I should kill them’. (said by Apihũ at Aifa 1967)
\end{xlist}
\end{exe}


%\begin{exe}
%	\ex \label{ex:eb}
%	\gll 
%	
%	\trans	
%\end{exe}



\section{Final observations}\label{s:eb5}

These many examples show how the several rich grammaticalized epistemic subsystems in the language may only be revealed in full through a discourse-centered approach that examines the natural contexts in which these occur.  A researcher depending primarily on elicitation or ordinary conversation might miss many of them altogether, as I discovered over time,  particularly as EM so often occur during socially discordant speech events that speakers are reluctant to invent for the foreign listener. 
Narrative and ritual speech frameworks include marking of changes in the intersubjective relation when new information is received; positive acceptance of a conjoint activity is requested by the interlocutor; and when acceptance or outright rejection of the interlocutor’s proposition occurs.  In a conversational context, there are a variety of graded ways these can be marked, from strong assertion to a weak, inferential marking and denial or disjoint marking.  Politeness or impoliteness occurs in many examples, particularly where the epistemic markers are used to mark judgments regarding conjoint activities.   With regard to a third person, the marker used depends on whether the speaker is making an inference of the third person’s stance based on common or historical memory (the logophroic \textit{kalaka} versus \textit{nipa}), as well as a stance marking based on conventional sociocultural knowledge regarding the behavior of a non-human actor (marked by \textit{nafa}).  
\cite{NuckollsLev2012}  note the importance of understanding sociocultural factors connected to the emergence of this complex type of linguistic feature.   Kalapalo EM occur in ordinary conversation, greetings, affinal civility (a register involving constrained speech and gestural modesty)  disagreement and even impolite description of a proposition ,  and most notably in quoted conversations that form important segments of extended narrative discourse.  EM contribute pragmatically in these contexts to the ‘unity and the experience of stance as it emerges in dialogic interaction’ (\citealt[35]{DuBois2007}), and there may be more person-to-person temporal and evidential features involved.  EM show types of distribution of responsibility, including requested or agreed-upon  alignment involving participation in an epistemic context,  as well as a speaker’s denial or rejection of co-participation, and denial of alleged responsibility.  Quoted speech in narrative discourse exemplifies a speaker’s need to shift between a focus on the epistemic object; marking of unique or shared perspectives, and reference to the interpersonal participation in epistemically situated discourses (which may also include deliberate impoliteness, involving the speaker’s  refusal to use the civility register as anticipated by interlocutors).  As is often the case, such contexts themselves change throughout a conversation, and there are a number of EM (especially sets C and D) that mark the nature of these person-to-person deictic shifts. This body of Kalapalo data forces us to look at interpersonal speech practice and experience,  which challenges a  simple contrast between private and public, micro- and macro-sociological events.  Memories of such dialogical production of meaning and the effect of these processes on social roles and relationships are preserved in the Kalapalo narrative materials discussed here, an important source for understanding the social and historical forces that have led to the emergence of this complex set of epistemic markers.

\section*{Phonological symbols}

Where my symbols are different from IPA conventions, the latter are placed in brackets. Alto Xingu conventions are marked (AX). \textit{Consonants}: voiceless stops: p: bilabial; t: post-alveolar; k: velar; k$^w$: dorso-velar; voiced stops: d: post-alveolar; g: velar; post-alveolar voiced stop: dy [d$y$]; prenasal voiced bilabial stop: mb [$^m$b]; voiced uvular flap: G; voiceless fricative: f [ɸ ]; voiceless alveolar fricative: s; voiceless glottal fricative: h; voiced post-alveolar affricate: ts; voiced velar affricate: č; bilabial nasal: m; velar nasal: n; post-alveolar nasal: ñ [ɲ]nh(AX) ; velar nasal: ŋ ng(AX);post-alveolar lateral: l; bilabial semivowel: w; velar semivowel: y. Vowels: front: medial:e; unrounded: i;  Central: medial: ï[ɨ]ü(AX); unrounded: a; Back: rounded: u; medial: o [ɤ]; stressed vowels show nasalization.
 
\section*{Abbreviations}
\begin{tabularx}{.45\textwidth}{lQ}

\textsc{adem} & adnominal demonstrative\\ 
\textsc{afr} & anaphoric focus referent\\ 
\textsc{aug} & augmentative\\ 
\textsc{cons} & consequential\\ 
\textsc{cum} & cumulative\\
\textsc{de} & de-ergative\\ 
\textsc{deo} & deontic\\ 
\textsc{des} & desiderative\\ 
\textsc{ds} & different subject\\ 
\textsc{eqa} & active equative\\ 
\textsc{eqs} & stative equative\\  
\textsc{ex} & existential\\ 
\textsc{exn} & existential nominalizer\\ 
\textsc{exp} & epistemic expressive\\ 
\textsc{fut}.1 & anticipated future\\ 
\textsc{fut}.2 & potential future\\ 
\textsc{idem} & identificational demonstrative\\ 
\textsc{imm} & immediate\\ 
\textsc{int} & intensive\\ 
\textsc{ipe} & end result of involuntary process\\  
\textsc{m} & modifier\\   
\textsc{nloc} & locative nominalizer\\ 
\textsc{nn} & negative nominalizer\\ 
\textsc{pdem} & pronominal demonstrative\\ 
\textsc{pe} & perpetual event\\  
\textsc{pej} & pejorative \\ 
\textsc{rq} & rhetorical question\\ 
\textsc{sd} & deontic subject\\ 
\textsc{se} & same event\\ 
\textsc{sel} & selective\\ 
\textsc{seq} & sequential\\ 
\textsc{ss} & same subject\\ 
\textsc{usin} & usuative  nominalizer\\ 
, & short pause\\ 
> &inverse marking\\
%... & \\
%... & \\
\end{tabularx}
%\begin{tabularx}{.45\textwidth}{lQ}
%... & \\
%... & \\
%\end{tabularx}

\section*{Acknowledgements}
%\citet{Nordhoff2018} is useful for compiling bibliographies.
My thanks go first to the organizers of this conference for inviting me to participate. and to the participants who kindly contributed comments and suggestions regarding my presentation.   Reviewers of my original draft provided many important suggestions for improving the description and analysis of this complex data,  though I take full responsibility for the final version.  Above all, I wish to express my unending gratitude to the three generations of Aifa and Tanguro communities in the Alto Xingu region of central Brazil.  As always I acknowledge their continuing to allow me to learn from them in so many different and lasting ways.  


\sloppy
\printbibliography[heading=subbibliography,notkeyword=this] 
\end{document}
